\documentclass[11pt, a4paper]{article}

\usepackage[slovak]{babel}
\usepackage[utf8]{inputenc}
\usepackage[T1]{fontenc}
\usepackage{times}
\usepackage[left=2cm, top=3cm, text={17cm, 24cm}]{geometry}
\usepackage[unicode, colorlinks, hypertexnames=false, citecolor=red]{hyperref}
\usepackage{tabto}

\begin{document}
	\begin{titlepage}
		\begin{center}
			\Huge
			\textsc{Vysoké učení technické v~Brně} \\
			\huge
			\textsc{Fakulta informačních technologií} \\
			\vspace{\stretch{0.382}}
			\LARGE
			Projektová dokumentácia k~predmetu IPK \\
			\Huge
			Packet sniffer (varianta ZETA)
			\vspace{\stretch{0.618}}
		\end{center}

		{\Large
			\today
			\hfill
			\begin{tabular}{r}
			Dominik Boboš (xbobos00)
			\end{tabular}
		}
	\end{titlepage}
	
	\tableofcontents
	\newpage
	
	\section{Úvod}
	Zadaním projektu bolo vytvorenie programu sieťového analyzátoru, ktorý zachytáva a filtruje na určitom sieťovom rozhraní prichádzajúce a odchádzajúce TCP a UDP pakety.
	
	Packet sniffer, nazývaný aj paket analyzátor pozostáva z dvoch hlavných častí. Prvou je sieťový adaptér, ktorý pripája program k existujúcej sieti. Druhá časť uskutočnňuje analýzu samotných zachytených paketov, v našom prípade zisťuje \emph(čas), kedy bol TCP alebo UDP paket zachytený, \emph{IP adresu} zdroja a cieľa a taktiež aj \emph{port} zdroja a 
	\section{Spustenie programu}
	\section{Implementácia}
	Program je implementovaný v jazyku \textbf{C} v súbore \texttt{ipk-sniffer.c}.  Je kompatibilný s linuxovými systémami a takisto aj so systémom macOS.
	\subsection{Použité knižnice}
	\subsection{Funkcie programu}
	Je rozdelený do niekoľkých funkcií. Implementácia sieťového adaptéru prípajajúceho sa na existujúcu sieť je v hlavnom tele \emph{int main(int argc, char *argv[])} využívajú funkcie knižnice \emph{pcap.h}.
	\section{Testovanie}
	\section{Použité zdroje}
	
	
\end{document}

	
	
	
